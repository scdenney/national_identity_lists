\documentclass[12pt]{article}
\usepackage{geometry}
\geometry{margin=1in}
\usepackage{graphicx}
\usepackage{amsmath, amssymb}
\usepackage{natbib}
\usepackage{setspace}
\usepackage[font=small,labelfont=bf]{caption}

\usepackage[hyperfootnotes=false]{hyperref}
%\usepackage{indentfirst}
%\setlength{\parindent}{15pt}  % Or any desired indent
\setlength{\parskip}{0pt}     % No extra space between paragraphs

\usepackage[style=apa,backend=biber]{biblatex}
\addbibresource{refs_new.bib}


% Keep DOI as a hyperlink
\DeclareFieldFormat{doi}{%
  \mkbibacro{DOI}\addcolon\space\nolinkurl{#1}}

% Simplify reference output
\AtEveryBibitem{%
  \clearlist{address}
  \clearfield{number}
  \clearfield{eid}
  \clearlist{location}
  \clearfield{month}
  \clearfield{issn}
  \clearfield{note}
  \clearlist{language}
  \clearfield{url}
}

% Customization for @misc entry
\DeclareBibliographyDriver{misc}{
  \usebibmacro{bibindex}%
  \usebibmacro{begentry}%
  \printnames{author}%
  \setunit{\addspace}%
  \printtext[parens]{\printfield{year}} % Prints the year in parentheses
  \setunit{\addperiod\space}%
  \printfield{title}%
  \setunit{\addspace}%
  \printfield{journaltitle}%
  \setunit{\addspace}%
  \iffieldundef{url}
    {}
    {\printfield{url}}%
  \usebibmacro{finentry}
}

\hypersetup{colorlinks=true,linkcolor=blue,citecolor=blue,urlcolor=blue}

\geometry{left=1.0in, right=1.0in, top=1in, bottom=1in}

\begin{document}

\begin{titlepage}
\title{Identity Conformity and Concealment in Taiwan and South Korea: Why Citizens in Divided Societies are Pressured to Overstate National Pride}
\author{
    Steven Denney\thanks{s.c.denney@hum.leidenuniv.nl} \\ Leiden University
    \and
    H. Christoph Steinhardt\thanks{hc.steinhardt@univie.ac.at} \\ University of Vienna
    \and
    Lisa Bhowmick\thanks{l.e.a.bhowmick@umail.leidenuniv.nl} \\ Leiden University
}
\date{}  % suppress date in \maketitle
\maketitle
\vspace{2mm}
\begin{center}
    \today
\end{center}



\singlespacing
\begin{abstract}
\noindent
To what extent do social and political pressures in divided societies compel individuals to exaggerate their allegiance to a dominant national identity? This study examines how identity conformity pressures shape expressions of national pride in Taiwan and South Korea, two similar democracies with divergent identity trajectories. Using list experiments to mitigate social desirability bias, we compare direct and indirect responses from subgroups with weak or contested national identities. We find evidence of pride inflation in both contexts, but considerably more in South Korea. In Taiwan, dual Taiwanese-Chinese identifiers modestly inflate Taiwanese pride but do not suppress Chinese identity, suggesting a pluralistic environment. South Koreans with weaker national attachment and especially North Korean migrants substantially overstate pride in South Korean identity and understate pride in their origin. The findings indicate a more consolidated identity regime in South Korea, where dominant narratives and institutional legacies generate stronger conformity pressures around national belonging.
\end{abstract}

\vspace{5mm}
\noindent\textbf{Keywords:} national identity, experiments, social desirability, Taiwan, South Korea\\

\end{titlepage}

% BEGIN MAIN TEXT
\doublespacing

\section*{Introduction}
Societies marked by territorial or political division often confront citizens with competing narratives of national identity. A large body of research has considered the processes of identity formation and competition, particularly in contexts marked by elite contestation, regime change, and external threat (e.g., Brubaker, 1996; Mylonas, 2012; Tudor, 2013; Zubrzycki, 2006; Abdelal et al., 2006; Wimmer, 2018; Ho, 2022). Yet, how contested settings generate social desirability pressures that reinforce dominant identity norms has received less attention.\footnote{Notable exceptions include Laitin (1998), Fouka et al. (2024), and Wu \& Lin (2024).} As Mylonas and Tudor (2021) note, much of the existing literature focuses on the origins and content of national identity and less so on how dominant identity narratives might change. This leaves open a theoretical and empirical gap about how identity norms are maintained, enforced, and negotiated by citizens.  

To address this gap, we examine Taiwan and South Korea – two structurally similar but identity-divergent societies that offer a comparative vantage on how identity norms operate under democratic conditions. Both countries share colonial legacies and histories of civil conflict, and each faces ongoing geopolitical tension tied to unresolved questions of sovereignty and national identity. Yet they differ significantly in the trajectories of identity formation and the regimes by which national identity is reproduced (Tu, 1996; Chu \& Lin, 2001; Hur, 2022; Hur \& Yeo, 2024). These environments provide an ideal setting to investigate how individuals navigate identity expression under social pressure, and how dominant identity norms play out. Prior work has shown that when individuals expect negative repercussions for holding particular views due to prevailing social norms, they may overstate or suppress those views (Jiang \& Yang, 2016; Valentim, 2024). This phenomenon, often termed preference falsification, has implications for understanding and measuring national identity, particularly in contexts where identity is contested.

To investigate how social expectations shape identity expression in contested settings, we conduct a comparative analysis of national pride in both countries. National pride serves as a powerful reflection of national identity, where individuals may feel pressure to conform to dominant narratives. We leverage Taiwan and South Korea’s shared historical and institutional features to better understand how variation in state-society relations and identity regimes conditions conformity pressures. We ask two primary research questions: First, do individuals in Taiwan and South Korea exaggerate their expressions of national pride? Second, do subpopulations with weak or dual identity backgrounds within each society overstate their pride more? Additionally, we explore a third question: Between Taiwan and South Korea, which national context exhibits stronger pressures to inflate national pride? To address these questions, we employ list experiments – an indirect measurement approach designed explicitly to mitigate social desirability biases inherent in self-reported attitudes. These experimental measures are complemented by observational indicators used to delineate subpopulations within each society.

Our findings indicate notable differences in the magnitude and distribution of conformity pressures between the two contexts, with South Korea exhibiting substantially greater pressures toward conformity than Taiwan. In Taiwan, individuals who weakly identify as Taiwanese – the dominant national identity – and those with a dual Taiwanese-Chinese identity overstate their national pride somewhat. However, all groups do not over- or understate their Chinese pride, indicating that expressing a dual or alternative identity remains socially permissible. In South Korea, conformity pressures are considerably more pronounced and pervasive, affecting segments of the majority population and minority groups. Native-born South Koreans with comparatively weaker attachments to the dominant national identity feel compelled to exaggerate their national pride, substantially more so than their Taiwanese equivalents. Questions for dual identity in South Korea were not asked because they would appear bizarre to native respondents. However, migrants from North Korea face intense social desirability pressures, leading them to dramatically inflate expressions of pride in South Korean identity while significantly understating their pride in North Korean heritage. 

These comparative insights highlight how social desirability biases significantly shape self-reported identities, complicating traditional understandings of national identity expression (Smith, 1993; Eun, 2020), particularly within divided societies. We interpret these dynamics as a form of nation-building through social conformity. In line with studies on social identity theory (Tajfel \& Turner, 1979; Scott, 1990), this illustrates how identity expression is not solely driven by personal conviction but also regulated by societal norms. South Korea’s cohesive national identity, deeply rooted in shared historical narratives and reinforced through close state–society linkages, generates strong social expectations for conformity (Lee \& Misco, 2014; Hur, 2022). Consequently, both North Korean migrants and less strongly identified native South Koreans are pressured to conform. Conversely, Taiwanese national identity remains more fluid and contingent. Democratization overthrew the previously dominant pan-Chinese identity regime and provided incentives to emphasize a pluralistic identity narrative (Tu, 1996; Chu \& Lin 2001; Wang \& Liu 2004). Although the question of identity is politically contentious (Ho 2022), a cultural attachment to Chinese identity is historically embedded. It does not carry the same degree of social stigma or ostracism observed for North Korean identity in South Korea.

This paper underscores how different historical and political trajectories shape contemporary norms surrounding national pride and identity expression, emphasizing that patriotism as a socially desirable attitude varies considerably even between otherwise comparable democracies (Fox \& Miller-Idriss, 2008; Hur, 2022). As both South Korea and Taiwan experience ultra-low fertility and increasing migration (Cheng 2020), understanding how national identity norms impact identity expression becomes increasingly important (Tsai, Tsai \& Huang, 2019; Denney, Ward \& Green, 2020; Rich et al., 2024; Everington, 2023; Shin, 2024) and widespread (Zhong, 2016).

The remainder of the paper proceeds as follows. The next section reviews select literature on national identity in our cases of choice and approaches used to measure politically sensitive questions. Next, we detail our data and methods, focusing on direct and list-based national pride measurements. We then present findings from the list experiments and highlight subgroup variations. Finally, we discuss broader implications for the study of national identity in divided societies, emphasizing how political and social pressures shape individuals’ public expressions of belonging.

\section*{Theoretical Foundations}
\subsection*{Patriotism, National Identity, and Identity Norms in Divided Societies}
National identity varies in content and salience. An important dimension of national identity is patriotism. It is both an individually held attitude and a social norm. Non-adherence to the norm can lead to social stigma. Norm compliance is strongly enforced by states in authoritarian regimes. Although enforcement is often more subtle in democracies, there are reasons to believe that identity norms are stricter and more actively enforced in divided societies.

National identity is constructed through shared narratives, institutional frameworks, and historical experiences (Anderson, 1983; Gellner, 1983; Brubaker, 1996). Contemporary scholarship treats national identity as a socially constructed and historically contingent phenomenon, shaped by elite narratives, mass incorporation, and state-led nation-building efforts (Mylonas \& Tudor, 2021). National identity varies in its content and salience across time and place, depending on institutional transmission, strategic elite framing, and individual-level experiences of inclusion or exclusion within the national community.

Closely related is patriotism, a positive affective attachment or pride in one’s country, distinguished from nationalism, which implies exclusionary attitudes or perceptions of national superiority (de Figueiredo \& Elkins, 2003; Kosterman \& Feshbach, 1989). National pride can also be specified by separating patriotism—positive sentiment toward a nation because of its economic or political accomplishments—from feelings of superiority over other nations, often labeled as chauvinism (Ha \& Jang, 2015; Gustavsson \& Stendahl, 2020).

Critically, national identity and patriotism function not only as individually held attitudes, but also as social norms that individuals feel compelled to uphold. Social identity theories emphasize that group memberships shape personal self-esteem and attitudes through the internalization of group values (Tajfel \& Turner, 1979). Yet individuals may also publicly embrace or perform certain identities primarily to comply with societal expectations rather than out of deep personal attachment (Cancian, 1975). Abdelal et al. (2006) describe such identities as governed by "constitutive norms," informal rules prescribing appropriate group behavior, making identity expression a form of norm compliance rather than personal conviction.

The enforcement of identity norms is capture in Noelle-Neumann’s (1974) ``spiral of silence” theory, which posits that individuals often suppress minority identities or opinions in public to avoid social isolation. This self-censorship reinforces dominant norms, generating a feedback loop in which socially desirable identities come to dominate public discourse, regardless of private beliefs. These dynamics are especially pronounced when minority identities carry social stigma, prompting individuals to conceal aspects of the self that are perceived as discrediting (Goffman, 1963; Suciu \& Culea, 2015). As Kuran (1995) argues, such concealment can lead to preference falsification, where private beliefs are deliberately misrepresented in public to conform to prevailing norms. In these contexts, expressions of national identity may reflect strategic adaptations to social pressure rather than sincere self-identification.

Empirical work in social psychology further substantiates this dynamic. Crandall et al. (2002), using their "justification-suppression model," illustrate that individuals regularly suppress genuine attitudes that conflict with prevailing social norms unless an acceptable justification for expressing these views is present. Extending this to identity, individuals privately ambivalent or weakly attached to dominant national identities might publicly display strong national pride when social contexts demand it. Conversely, when cosmopolitan or subnational identities become normatively valued, individuals may publicly downplay nationalist sentiments to maintain social acceptance (Fouka et al., 2024). Thus, national identity and patriotism often reflect a tension between internal attitudes and external normative pressures, leading individuals to strategically calibrate their public identity performances based on perceptions of social desirability.

The enforcement of national identity norms takes different forms depending on regime type. In authoritarian regimes, state power directly dictates which identities are acceptable, compelling individuals to display loyalty (Wedeen, 1999; Dukalskis, 2020). National pride is not just encouraged but mandatory, with severe consequences for noncompliance. In China, the government actively uses patriotic propaganda to promote the dominant mode of national identity and loyalty to the regime (Wang, 2012; Biberoğlu, 2022). Scott’s (1990) concept of "hidden transcripts" highlights how individuals in such settings maintain private dissent while publicly conforming.

Democratic societies enforce national identity through subtler social mechanisms. Fox and Miller-Idriss (2008) argue that nationalism in democracies is reproduced through daily practices, such as standing for the national anthem, expressing pride in national history, and adhering to civic rituals. This kind of banal nationalism (Bellig, 1995), while also relevant in autocratic regimes, is particularly important in democracies since the availability of political competition enables the boundaries of national identity to be continuously renegotiated through these daily practices (Goode, 2021). Although individuals in democracies are not formally obligated or overtly expected to express national pride, they may nonetheless incur reputational costs for failing to do so. Vlachová (2019), examining survey responses in Czechia, finds that national pride can be so deeply embedded in social expectations that individuals overstate their patriotic sentiments to avoid appearing disloyal. By contrast, West Germans understate their desire to leave behind a “perpetrator focused narrative” of German history for fear of ostracism (Fouka et al., 2024).  

The enforcement of national identity norms is especially salient in divided societies, which we understand as contexts where national belonging is contested through mutually exclusive and politically irreconcilable identity narratives. These divisions go beyond routine pluralism or identity diversity; they involve rival claims to legitimate nationhood, often rooted in unresolved conflicts over territory, sovereignty, or statehood (Dittmer, 2006; Kachuyevski \& Olesker, 2014). In such environments, the act of self-identification becomes not only politically charged but socially consequential, as individuals may face reputational costs or even exclusion for expressing identities that diverge from dominant narratives (Mac Ginty, 2017). Identity norms in these settings are enforced not only by formal institutions but also through community policing and public discourse, with deviance incurring social or political penalties (Kachuyevski \& Olesker, 2014). These pressures should be most pronounced in societies where identity conflict is linked to civil war, secession, or militarized confrontation, as in the cases of South Korea and Taiwan. Given these considerations, Taiwan and South Korea are “most likely” cases for identity-based preference falsification.

\subsection*{National Identity Norms in Taiwan and South Korea}
Taiwan and South Korea share significant historical and structural similarities. However, their national identity formation and current states of identity contestation diverge considerably. This may have important consequences for the strength of national identity norms.

Taiwan and South Korea are high-income societies of comparable size, with export-led economies and a Confucian cultural background. Both are characterized by ultra-low fertility and an increasing dependence on immigration (Cheng, 2020). They share historical experiences of Japanese colonial rule and Cold War authoritarianism, and both transitioned to democracy during the third wave of democratization in the 1980s (Tsai, 2009; Hur \& Yeo, 2024). Today, each faces geopolitical pressure from authoritarian neighbors (China and North Korea) as well as challenges from competing concepts of national identity. However, their national identity formation has diverged considerably.

Taiwan’s identity regime has undergone dramatic change. Under Kuomintang (KMT) rule via martial law (1949–1987), a pan-Chinese national identity was imposed through top-down mechanisms, including Mandarin language imposition, historical education, and the suppression of dissent (Chu \& Lin 2001). Adherence to Chinese nationalism was quite successfully entrenched (Yang 2007). However, Taiwanese language and identity remained entrenched among lower classes, and a clandestine Taiwanese identity movement became a critical driving force of democratization (Chu \& Lin 2001; Yang 2007; Ho 2022), and remained closely intertwined with partisan polarization ever since (Qi, 2016; Qi \& Lin, 2021). Public expressions of Taiwanese identity became, first, politically permissible and dominant. Longitudinal data shows that the proportion of citizens identifying as exclusively Taiwanese rose from under 20 percent in the early 1990s to over 60 percent by the 2010s (National Chengchi University, 2025). 

However, although identity politics figures prominently in Taiwan, democratization also created incentives for a more inclusive identity narrative. Politicians sought to replace the sub-ethnic division between native Taiwanese and Mainlanders with an inclusive narrative to maximize votes (Wang \& Liu 2004). Even as elite narratives de-emphasized subethnic cleavages, long-standing distinctions between native Taiwanese (Hokkien/Hakka—\textit{Benshengren}) and Mainlander-descent groups (\textit{Waishengren}) continue to orient social expectations about belonging. In such settings, the modal-origin majority (\textit{Benshengren}) faces stronger normative expectations to perform national pride than minority-origin groups (\textit{Waishengren}).\footnote{In Taiwan, the terms \textit{Benshengren} (literally “people of this province”) and \textit{Waishengren} (literally “people from outside the province”) denote historically salient sub-ethnic categories.} A new school curriculum emphasized not only Taiwan’s distinctiveness from China but also its multiethnic heritage (Wang \& Liu, 2004; Chen, Lin, \& Yang, 2023). As a result, whether one identifies as Taiwanese or Chinese increasingly functions as a political identifier rather than a purely ethnic or cultural one (Zhong, 2016). Majorities did not relinquish their culturally Chinese identity (Zhong, 2016), and significant minorities maintain China-friendly attitudes (Wu \& Lin, 2024).

How far this relative pluralism translates into social-desirability pressures remains an empirical question. Recent evidence indicates that some pressure is tangible, as Taiwanese understate China-friendly opinions to align with prevailing norms (Wu \& Lin, 2024). At the individual level, citizens who hold both Taiwanese and Chinese attachments occupy a boundary position: their public affirmations of Taiwanese pride are especially visible and contestable in everyday interaction, making them natural focal points for selective enforcement of identity norms in a pluralist regime. Emerging work also shows that public attitudes toward immigrants in Taiwan vary systematically with racialized appearance and local linguistic proficiency, underscoring how identity boundaries are policed in everyday life (Rich et al., 2022; Kao \& Liu, 2025).

South Korea, by contrast, has maintained a stable and cohesive national identity grounded in ethnic nationalism and anti-communism, institutionalized during military rule and reinforced – not destabilized – through democratization (Lee \& Misco, 2014; Lee, 2021). One might ask whether South Korea qualifies as a divided society, given that North Korea’s competing national identity project has largely been treated as external and illegitimate, never enjoying substantial support within the South. However, the enduring military and ideological threat posed by the North continues to shape South Korea’s identity regime by reinforcing dominant narratives around loyalty, unity, and state legitimacy.\footnote{In the 1980s and 1990s, segments of the South Korean student movement – particularly the \textit{Jusapa} (pro–Juche faction), the National Liberation (NL) faction, and \textit{Hanchongnyon} (South Korean Federation of University Students Councils) – embraced elements of North Korean ideological thought and promoted unification largely on Northern terms (Park, 2002). Although never truly mainstream, these factions became politically visible and provoked state responses that contributed to the broader securitization of national identity in South Korea. Their activities intersected with a national identity regime already shaped by Cold War confrontation and ideological boundary policing (Shin, 2006). The continued enforcement of the National Security Act and its repeated affirmation by South Korea’s Constitutional Court reflect the state’s view that pro-North sentiment, however marginal, remains a challenge to South Korean sovereignty and national cohesion (redacted).} South Korea’s framing of national pride as a civic duty, reinforced through educational institutions, historical narratives, and state rituals (Hur, 2022; Lee, 2023), can be understood in part as a response to North Korea’s competing identity project.

These norms continue to shape behavior today. Marginalized populations such as North Korean migrants must publicly affirm loyalty to South Korea to access full membership, which is a dynamic that Hough and Bell (2020) describe as conditional inclusion. Kim and Oh (2001) find that defectors often display a disconnect between their implicit and explicit identities, while [redacted; Kim \& Lee (2023)] show that native South Koreans themselves exhibit social desirability bias in their attitudes toward these migrants. Together, these findings illustrate how national identity norms in South Korea are upheld through a combination of public performance and reputational pressure.

The contrasting trajectories of identity formation in Taiwan and South Korea allow us to examine how preference falsification and social desirability pressures vary with the configuration of national identity norms. In line with the logic of a \textit{most similar systems design} (MSSD) (Seawright \& Gerring, 2008), the two cases share key structural and historical features, yet differ on a crucial dimension: the degree to which national identity is contested and internally coherent (Abdelal et al., 2006). South Korea’s enduring sense of ethnic unity has fostered stronger state-society integration than Taiwan, producing a cohesive and deeply embedded national identity (Hur, 2022). In Taiwan, the process of democratization overthrew the existing Chinese identity regime and left Taiwanese national identity comparatively more fluid. This divergence may have important implications for how individuals express national pride:in South Korea, individuals may experience greater pressure to align their public expressions with dominant norms than in the more pluralistic environment of Taiwan. 

Holding institutions and measurement constant, we expect the pattern of discrepancy in how national pride is expressed to map onto each case’s identity regime. This sets up the hypotheses that follow.

\subsection*{Hypotheses}

% --- Framing for H1–H2: general, theory-driven expectations (use "case/society") ---
Our theoretical framework yields two general expectations. First, we posit a \emph{common within-case mechanism} operating through subjective identity strength. Individuals who are less securely attached to the dominant national identity should be more prone to inflate direct self-reports when signaling to others. Second, leveraging our most-similar-systems design, we anticipate that the \emph{magnitude and distribution} of this mechanism will differ across cases in ways that reflect their identity regimes.

\begin{quote}
\textbf{$H_1$ (Within-case, subjective identity strength).} In both societies, individuals with weaker identification with the dominant national identity will exhibit larger discrepancies between public self-reports of pride and less obtrusive measures than individuals with stronger identification. This establishes the common mechanism of conformity pressures.
\end{quote}

\begin{quote}
\textbf{$H_2$ (Cross-case magnitude, subjective channel).} For comparable groups (all respondents, weak identifiers, strong identifiers), the conformity gaps will be larger in \emph{South Korea} than in \emph{Taiwan}. This is the central cross-case implication of the MSSD.
\end{quote}

% --- Bridge from general to country-specific implications, organized by channels ---
Having stated the general expectations, we now identify \emph{who} bears conformity pressure in each case, distinguishing the \emph{subjective} channel (identity strength) from the \emph{objective/ascribed} channel (origin/background). The two cases differ in the extent to which norms are enforced and which positions attract scrutiny.

\begin{quote}
\textbf{$H_3$ (Taiwan: selective, identity–contingent enforcement).} Taiwan’s pluralistic regime concentrates enforcement at identity boundaries. 
\emph{Subjective channel:} Dual Taiwanese–Chinese identifiers are expected to overstate Taiwanese pride more than Taiwanese–only identifiers do (\emph{Dual} $>$ \emph{TW–only}). Dual identifiers occupy a legitimate yet publicly ambiguous position that invites audience scrutiny and leaves room to credibly overstate pride. Taiwanese–only respondents already align with the dominant norm and thus have less scope or reason to inflate.
\emph{Objective/ascribed channel:} Conformity pressures should be detectable for the most common origin group but attenuated for minority-origin groups: Benshengren should show a noticeable gap, whereas Waishengren should show a small or non-significant gap (\emph{Ben} $\gtrsim$ \emph{Wai}). In Taiwan, it is socially acceptable for minority-origin respondents to be less proud; the majority group does not impose strong upward pressure on them to perform pride.
%% Robustness (Taiwan): a secure/insecure recode should reproduce the selective pattern while increasing cell sizes; full Ns/CIs reported in the Appendix.
\end{quote}

\begin{quote}
\textbf{$H_4$ (South Korea: broad, graded enforcement).} South Korea’s less pluralism and more consolidated identity regime enforces identity norms broadly, with pressure graded by vulnerability and audience scrutiny. 
\emph{Subjective channel:} Among natives, weaker identifiers should display larger gaps than strong identifiers (\emph{Weak natives} $>$ \emph{Strong natives}). 
\emph{Objective/ascribed channel:} North Korean–origin migrants should exhibit the largest gap, exceeding native groups, yielding a graded ordering \emph{NK-origin} $>$ \emph{Weak natives} $>$ \emph{Strong natives}. The migrant case reflects intense scrutiny and conditional inclusion in a consolidated identity regime and is not directly comparable to Taiwan’s Waishengren.
%% Note: NK-origin migrants are analytically distinct; we treat them as a case-specific objective/ascribed test rather than a cross-case analogue to Wai.
\end{quote}

 We now turn to the data and methods used to evaluate these hypotheses. To assess such hidden conformity pressures, we turn to indirect measurement techniques designed to mitigate social desirability bias.

\section*{Data and Methods}
When asked about national pride, respondents frequently adjust their answers to align with perceived social expectations, a phenomenon known as social desirability bias (Knoll, 2013; Meitinger, 2018). Particularly in politically sensitive or stigmatized contexts, such as national identity in divided societies, this bias can result in significant misrepresentation or preference falsification of individuals' true sentiments (Jiang \& Yang, 2016; Kuran, 1995; Valentim, 2024).

Indirect survey methods, such as list experiments (also known as item-count techniques), are useful for accurately measuring national pride and identity in contexts prone to such biases. List experiments enable respondents to anonymously express sensitive views by embedding controversial items among innocuous ones, thereby significantly reducing social desirability bias (Kuklinski, Cobb, \& Gilens, 1997; Chapkovski \& Schaub, 2022; Karpowitz et al., 2023). These techniques help identify concealed attitudes across various sensitive contexts, including nativism and racial prejudice, as well as authoritarian support and fraudulent election practices (Çarkoğlu \& Aytaç, 2015; Moseson et al., 2015; Lax, Phillips \& Stollwerk, 2016).

Our study employs list experiments to obtain accurate measures of national pride and identity, minimizing biases associated with direct questioning. Recognizing the pronounced susceptibility of national pride and identity expressions to social desirability pressures, particularly in politically polarized and divided societies, we adopt this indirect approach for our study of Taiwan and South Korea.

We examine national pride across three groups: Taiwanese respondents, native South Koreans, and South Korean citizens of North Korean origin.\footnote{People of North Korean origin residing in South Korea refers to individuals who were born in or formerly resided in North Korea and later migrated to South Korea, whereby the were granted citizenship as per the country's constitution. These individuals are often referred to as "North Korean defectors" (\textit{talbukja}), though the term "North Korean refugees," "North Korean migrants," or "migrants of North Korean origin" are sometimes used to emphasize their resettlement experience and reduce the stigma associated with defection. We use a combination of labels emphasizing their migrant background and origins.} By employing both direct questioning and list experiments, we compare responses to test for preference falsification and better understand how individuals align their expressed national identities with perceived social expectations.

We conducted surveys with a total of 2,050 respondents in Taiwan, 1,994 respondents in South Korea, and 301 respondents of North Korean origin currently living in South Korea between December 2023 and January 2024. We recruited respondents for the Taiwanese and native South Korean samples via Qualtrics online panels, using quotas for age, sex, and region to approximate representativeness. We collected data from migrants of North Korean origin who spent no fewer than 12 years in North Korea through Woorion, a South Korea-based NGO specializing in services for North Korean resettlers. Appendix A of the Supplementary Information (SI) contains additional details on recruitment, sampling strategies, and the composition of each sample.

In all surveys, respondents completed the list-experiment blocks first, followed by the direct pride questions. This ordering minimizes demand/priming from explicit identity items before the indirect measures.

We then posed direct questions about national pride. Taiwanese respondents indicated their agreement with statements such as, “I am proud to be Taiwanese” and “I am proud to be Chinese” (using \textit{zhongguo ren} to specify Chinese identity\footnote{\textit{Zhongguo ren} literally translates to “Chinese person”. It was used in the KMT’s pan-Chinese identity project under martial law and has become increasingly associated with the People’s Republic of China. In contrast, references to the “Chinese nation” (\textit{zhonghua minzu}) carry a cultural connotation of Chinese heritage. \textit{Zhongguo ren} is used in standard self-identification questions and is more politically contested. It is therefore our term of choice here.}). South Koreans and migrants of North Korean origin responded to a similar format, focusing on pride in being “South Korean.” We did not ask South Koreans about their pride in being North Korean, as that would be regarded as wholly unrealistic if not outright bizarre in the South Korean context. This is a testament to the stability of the national identity we described above. We did, however, query North Korea migrants about their pride in being “from North Korea”.\footnote{Furthermore, regarding question wording for North Korean migrants, we initially considered phrasing the item as “pride in being North Korean”, but our NGO partner (see Appendix A) strongly advised against this wording, citing concerns that it could be politically sensitive or alienating for respondents. In response, we adopted the softer phrasing “pride in being from North Korea,” which was perceived as less ideologically loaded and more acceptable within the migrant community.} These direct items allowed us to compare self-reported pride with estimates derived from an indirect approach.

Our primary outcomes for hypothesis testing are pride in being \emph{Taiwanese} (Taiwan) and pride in being \emph{South Korean} (South Korea), corresponding with hypotheses $H_1$–$H_4$. The alternative identity items for pride in being \emph{Chinese} (Taiwan) and, for North Korea-origin migrants, pride in being \emph{from North Korea} are used as auxiliary outcomes to diagnose the direction of enforcement (inflation vs. suppression). In Taiwan, we expect little to no suppression of \emph{Chinese} pride (gaps near zero, including among dual identifiers), consistent with selective enforcement targeting \emph{inflation} of Taiwanese pride rather than suppression of Chinese pride. Among North Korean migrants in South Korea, we expect \emph{suppression of origin pride}, i.e., a positive gap for “proud to be from North Korea” (\emph{List $>$ Direct}), mirroring the inflation observed for South Korean pride.

List experiments were used to detect social desirability bias in responses. Each participant received a set of mundane, non-sensitive statements (e.g., “I believe hard work is important to success” or “I like teamwork”), which serve as baseline items to mask individual endorsement of the sensitive item. We added one sensitive item – such as “I am proud to be Taiwanese”, “I am proud to be Chinese”, “I am proud to be South Korean”, or “I am proud to be from North Korea” – randomly assigned as a treatment statement. Participants indicated only how many statements they endorsed in total, without specifying which ones. By comparing the average endorsement count between the treatment group (which received the sensitive item) and the control group (which did not), we can estimate the proportion of respondents genuinely endorsing the sensitive item while reducing social desirability bias.

The list experiment approach reduces bias because individuals need not admit or deny the sensitive attitude directly. We tailored the items for Taiwan and South Korea to capture pride in being Taiwanese or Chinese and pride in being South Korean, respectively, while ensuring consistency in structure. In both surveys, the list blocks preceded the direct blocks by design. These efforts maintained comparability across the two national settings. In Appendix B of the SI, we provide the full text of the direct questions and lists and describe how we distributed sensitive items across subgroups. Appendix C reports balance test output to confirm successful and balanced randomization.

We structured our analysis first to examine how the strength of individuals' identification with the dominant national identity shapes their likelihood to overstate pride, before focusing specifically on subgroups where we expect pronounced pressures for conformity or concealment. This ordering allows us to initially establish baseline differences in how social desirability bias operates across groups distinguished by identity strength. Subsequently, we investigate particular subgroups that theoretically embody contested or stigmatized identities, where social pressures to conform or conceal are expected to be especially acute.

For the analysis that supports direct comparison between the two cases ($H_1$–$H_2$), we dichotomize identity strength on the 0–10 scale such that “strong” identifiers are those who select 10 (the top-box category) and “weak” identifiers are those who select any value below 10. This measurement strategy aligns with our theoretical expectations that respondents with prior strong national identities should experience less pressure to artificially inflate their pride, whereas those with weaker identification might be particularly susceptible to conformity pressures and thus more prone to overstating their national pride. Importantly, our interest here is not primarily in absolute differences in pride across these groups, as higher reported pride among strong identifiers is anticipated; rather, it is in how the magnitude of social desirability pressures varies based on pre-existing identity commitments within and between the two cases.

After establishing these baseline patterns, we turn to examine specific subnational groups within each country ($H_3$-$H_4$). Specifically, we analyze the distinct experiences of "Taiwanese-only" identifiers and dual "Taiwanese-Chinese" identifiers in Taiwan, and South Koreans of North Korean origin in South Korea. It is important to emphasize that these groups are not directly comparable. For instance, due to their marginalized position in society, North Korean migrants should be more prone to social pressure than dual identifiers in Taiwan. Nonetheless, the groups represent theoretically meaningful cases for testing how social conformity pressures operate under conditions of contested identity, particularly in contexts where identity boundaries are actively negotiated and often politically charged. By investigating these specific subgroups, we deepen our understanding of how identity conformity varies within societies that face differing historical legacies and contemporary political contexts.

To facilitate the additional subgroup analysis in Taiwan, we measured Taiwanese and Chinese identity separately using two distinct 0–10 scales rather than a single categorical measure. This approach allowed us to capture nuanced gradations in identity strength (Levy 2014; Steinhardt, Li, and Jiang 2018). Respondents rated their level of identification with each identity. We defined those strongly identifying as Taiwanese as respondents scoring at the 75th percentile or higher (a rating of 10), reflecting a clear and intense identification. Similarly, we identified respondents as strong Chinese identifiers if they rated their Chinese identity at five or above (also the 75th percentile). Using these thresholds, we created two subgroups: "Taiwanese-only" identifiers, who report strong Taiwanese identity but weak Chinese identity (below the specified threshold), and "Taiwanese-Chinese" identifiers, who simultaneously express meaningful identification with both identities. Given the negligible presence of respondents holding exclusively strong Chinese identification, this dual categorization effectively captures the identity dimensions relevant for our analysis. Appendix D of the Supplementary Information provides more detail on the coding procedures and thresholds for each subgroup.

In addition to these main subgroups, we also consider an additional subgroup measure in Taiwan. We distinguish Benshengren (local Hokkien or Hakka paternal origin) from Waishengren (Mainlander paternal origin), following standard practice in Taiwan surveys. This subgroup split captures historically salient identity backgrounds and allows us to assess whether conformity pressures are stronger for the modal group (Benshengren) than for minority-origin groups (Waishengren). We code paternal origin as the basis for classification: Benshengren include respondents of local Hokkien or Hakka descent, while Waishengren include respondents of Mainlander descent who arrived with or after the Kuomintang in the late 1940s. These categories reflect enduring historical cleavages and are widely used in Taiwanese social science, though their salience has diminished over time. Appendix D in the SI reviews Taiwanese subgroup distributions.

For North Korean migrants in South Korea, we measured identity and pride separately, using the same 0–10 scale method described previously. Migrants rated how strongly they identify as South Korean and how strongly they identify as North Korean. To facilitate subgroup comparisons, we also applied a 75th percentile cutoff to distinguish between stronger and weaker identifiers, consistent with the approach used in Taiwan. Given this group’s unique status, originating from a stigmatized and contested identity within South Korean society, we directly compared their pride in each identity. By evaluating both direct responses and list-experiment estimates, we capture the extent to which North Korean migrants feel compelled to overstate pride in the South Korean identity or understate pride in their North Korean heritage. This measurement strategy enables us to assess the strength and direction of social desirability pressures specific to this subgroup.

Beyond the pride outcomes used for $H_1$–$H_4$, we fielded additional list-experiment items to test the \emph{scope} of enforcement beyond identity performance (pride) and to locate where social monitoring is most active. In Taiwan, we look at support for a formal declaration of independence\footnote{In the Taiwanese context, “formal declaration of independence” refers to a \emph{de jure} statement that Taiwan (Republic of China) is an independent country distinct from the People’s Republic of China. We deliberately frame this as a policy position (not an identity affirmation) and avoid partisan cues or referendum wording. Because the status quo is widely understood as \emph{de facto} independence, the declaration is rhetorically salient yet contested: some view it as the symbolic completion of statehood, while others see escalation risk. These cross-cutting considerations make it a useful scope test for conformity pressures, as any direct–list gap could plausibly run in either direction.} In South Korea, we consider support for the National Security Act (NSA) and views on unification. The independence item probes whether selective enforcement targets signaling of Taiwanese identity rather than a sovereignty-related declaration. In South Korea, the NSA item captures loyalty-inflected security norms in a more identity-consolidated regime, and the unification item gauges whether a once-central symbol is still enforced. The Taiwan items were fielded in the same instrument as the pride measures (within-sample comparability). The South Korea items come from a separate 2021 online panel with Rakuten Insight, providing an out-of-sample robustness probe of norm-sensitive items. Estimation mirrors the main analyses (list = linear; direct = logistic; gap = \emph{List $-$ Direct}).

Across all models, we used regression models to estimate indirect endorsements of sensitive items, employing the \texttt{ictreg} package in R, which supports a linear approach for estimating underlying endorsement rates. Each list-experiment model included covariates for statistical controls, including age, sex, university education, and partnership, to adjust for potential confounding. For the South Korean citizens of North Korean origin models, we included controls for length of time spent in North Korea prior to leaving and the number of years spent in South Korea instead of an age variable. Educational attainment is measured by whether respondents completed a university-level education in North Korea. For the corresponding direct measures, we used logistic regression (\texttt{glm} with a binomial family) to predict the likelihood of openly endorsing national pride, including the same control variables. We then combined each indirect (list) model with its corresponding direct model to compute average predicted endorsement rates for each group. We define the conformity gap as \emph{List $-$ Direct} (percentage points). Separate models were estimated for different subgroups – high versus low national identification groups; Taiwanese-only and Taiwanese-Chinese identifiers in Taiwan; native South Koreans and North Korean migrants in South Korea – enabling subgroup-specific analyses of preference falsification. We report average predicted percentages with 90\% and 95\% confidence intervals.

A final data note concerns Taiwan’s “New Residents” (\textit{xin zhumin}), a population of foreign-born immigrants (primarily from Mainland China and Southeast Asia) whose presence has grown in recent decades and who could in some respects be considered analogous to North Korean migrants in South Korea. According to official statistics, foreign residents made up about 4\% of Taiwan’s population in late 2024 (NIA, 2024), but in our survey, they constituted only around 1\% of respondents. This small sample size precludes reliable list-experiment estimation, given the variance-intensive nature of the method.\footnote{To address concerns, we re-estimated all models excluding immigrant respondents as a check on results; they are completely unchanged.}

\section*{Findings}
Our list experiment results present evidence for systematic patterns of overstated national pride shaped by identity norms and social desirability pressures in Taiwan and South Korea. Figures 1-4 illustrate the comparative analyses, showing the proportions of respondents expressing pride in their national identity through direct and indirect estimates. 

By highlighting discrepancies between these direct and indirect estimates, we identify the presence and extent of inflated or suppressed expressions of national pride. In Taiwan, we observe nuanced patterns of identity conformity. Individuals with weaker attachment to the dominant Taiwanese identity experience considerable pressure to exaggerate their pride publicly. In contrast, those with a strong Taiwanese identity express consistent pride regardless of the measurement method. Notably, we find no comparable pressure on Taiwanese citizens to suppress expressions of pride in a Chinese identity, suggesting a tolerant and pluralistic identity norm environment that accommodates dual identification. 

South Korea shows some similar patterns, but it demonstrates more pronounced and uniform pressures to conform, compelling even majority populations with weaker national identification and marginalized migrants from North Korea to significantly overstate their allegiance to the dominant national identity while understating pride in any alternative identity. These findings underscore how divergent state-society relationships and national identity frameworks across two otherwise comparable democracies drive distinct preference falsification and conformity mechanisms.

\subsection*{Identity Strength and National Pride}
Figure 1 presents a comparative analysis of national pride based on identity strength among Taiwanese and South Korean respondents. This analysis, which compares direct (self-reported) estimates with list-experiment measures, provides the most directly comparable analysis across the two national populations.
In the aggregate (all respondents), both Taiwanese and South Koreans express similarly high levels of national pride in direct questioning (~79\%). However, the indirect (list-experiment) measures show lower levels of pride (i.e., the “true” preference) – approximately 67 percent for Taiwanese and about 59 percent for South Koreans. This yields significant differences of 12.4 percentage points (pp) for Taiwanese and 20.7pp for South Koreans, suggesting that both groups inflate expressions of pride, albeit more substantially in South Korea.

Breaking respondents down by identity strength, strong national identifiers in both countries report high direct levels of national pride (86.1\% for Taiwanese and 88.5\% for South Koreans), as expected. Indirect measures remain high but somewhat lower (77.1\% Taiwanese and 76.6\% South Koreans), with statistically insignificant differences (~9 to 12pp). This suggests that individuals with strong identifiers, already deeply rooted in their national identity, face less pressure to exaggerate their pride.

However, respondents with weak national identity demonstrate pronounced social desirability bias. Weak national identifiers in Taiwan report pride at 70.4 percent, but this drops substantially to less than 54 percent when using indirect measures, constituting a difference of 16.6pp. South Korean weak identifiers exhibit an even larger disparity: 70.4 percent for the direct measure versus 43.9 percent via the list, representing a substantial difference of ~27pp. These findings underscore that individuals with weaker national attachments experience stronger pressures to publicly conform to national pride norms, especially in South Korea.

%%% FIGURE 1 HERE
\begin{figure}[H]
    \centering
    \includegraphics[width=1\textwidth]{Figure 1.pdf}
    \caption{National Pride by Identity Strength in Taiwan and South Korea
    \vspace{1.5mm} \newline {\small \emph{Note}: Figure shows estimated proportions of respondents in Taiwan and South Korea who express pride in their national identity, based on both direct (self-reported) and list-experiment measures. Results are presented for all respondents as well as for subgroups defined by strong versus weak identification with the national identity. List estimates are obtained using linear probability models, while direct responses are modeled with logistic regression. Estimates control for age, sex, education, and political identification. Error bars represent 90\% and 95\% confidence intervals. The “difference” column captures the gap between list and direct responses, reflecting potential social desirability pressures. Results suggest broader and stronger inflation in South Korea across both weak and strong identifiers, whereas in Taiwan inflation is more selective and concentrated among weak identifiers.}}
\end{figure}

\newpage

\subsection*{Subnational Group Focus: Taiwan}
Figure 2 examines Taiwanese pride among three subgroups: all respondents, Taiwanese-only identifiers, and Taiwanese-Chinese identifiers. For all respondents, the direct measure of Taiwanese pride stands at approximately 79 percent, whereas the list-experiment estimate is 67 percent. The difference of -12pp (the list estimate subtracted from the direct question estimate), significant at the 95 percent confidence level, suggests that a portion of the broader Taiwanese population may feel some social expectation to emphasize national pride when asked directly.

Looking specifically at Taiwanese-only identifiers, their direct (~89\%) and list (82\%) measures show a discrepancy of approximately –6.5pp. The small difference is statistically insignificant at the 95 and 90 percent confidence level, indicating that these individuals hold considerable pride in being Taiwanese, absent social desirability bias. Their comfort in asserting this identity implies that expressing Taiwanese pride is socially accepted for this subgroup, leaving less reason to falsify their attitudes.

By contrast, Taiwanese-Chinese identifiers present a relatively large disparity between direct (70\%) and list (49\%) measures. The difference in direct point estimates between the two groups is as expected – those with mixed identities are less likely to show pride in being Taiwanese. However, the focus here is on social desirability bias.\footnote{It is, of course, possible that identity norms lead those with weaker Taiwanese identities to also overstate their perceived identity strength here, while understating their Chinese-ness. This would in effect “dilute” our “Taiwanese only” group with respondents who have a weaker identity than they admit. This makes us more confident that there is no significant preference falsification in this group.} The difference of –21pp is substantial. The indirect results suggest that individuals with a partial Chinese identity experience strong social pressures to conform to prevailing norms that emphasize a Taiwanese national identity. Consequently, they are more likely to overstate Taiwanese pride when asked directly.

We next examine Taiwanese pride by paternal-origin background. Figure 3 reports estimates for \textit{Benshengren} (Hokkien/Hakka) and \textit{Waishengren} (Mainlander descent). As expected, Waishengren report lower direct Taiwanese pride overall (67.6\%) than Benshengren (81.3\%). More notable is the pattern in conformity gaps: Benshengren show a sizable, statistically significant gap between direct and list estimates, whereas Waishengren show a smaller, statistically non-significant gap.

At first glance, the smaller Waishengren gap may seem counterintuitive if one expects majority pressure to be applied to minorities. Our interpretation, consistent with $H_3$’s selective, identity-contingent enforcement, is that \textit{pressure is subgroup-specific}: members of the modal-origin majority (\textit{Benshengren}) carry stronger normative expectations to perform Taiwanese pride. Among them, weaker identifiers have both motive and opportunity to overstate pride in direct answers, producing the observed gap. By contrast, minority-origin respondents (\textit{Waishengren}) face lower expectations to perform the dominant identity; their lower direct levels are unsurprising, and with less social scrutiny—and less credible payoff to overstatement—their gap is smaller and not distinguishable from zero. This pattern sharpens the selective-enforcement claim: the burden to signal rests primarily with the majority background group, not with minorities.

%%% FIGURE 2 HERE
\begin{figure}[H]
    \centering
    \includegraphics[width=1\textwidth]{Figure 2.pdf}
    \caption{Pride in Being Taiwanese by Identity Subgroups I
    \vspace{1.5mm} \newline {\small \emph{Note}: Figure shows estimated proportions of respondents in Taiwan who express pride in being Taiwanese, using both direct (self-reported) and list-experiment measures. Respondents are grouped into three identity categories: all respondents, Taiwanese-only identifiers, and Taiwanese-Chinese identifiers. List estimates are based on linear probability models; direct responses are modeled using logistic regression. Estimates control for age, sex, education, and political identification. Error bars represent 90\% and 95\% confidence intervals. The “difference” column reflects the gap between list and direct responses and captures potential social desirability bias. Results suggest that while Taiwanese-only identifiers express pride consistently across measures, Taiwanese-Chinese identifiers show substantial inflation in direct responses, indicating stronger conformity pressures within this subgroup.}}
\end{figure}

\pagebreak

%%% FIGURE 3 HERE
\begin{figure}[H]
    \centering
    \includegraphics[width=1\textwidth]{Figure 3.pdf}
    \caption{Pride in Being Taiwanese by Identity Subgroups II
    \vspace{1.5mm} \newline {\small \emph{Note}: Figure shows estimated proportions of respondents in Taiwan who express pride in being Taiwanese, using both direct (self-reported) and list-experiment measures. Respondents are grouped by paternal-origin background: \textit{Benshengren} (Hokkien/Hakka) and \textit{Waishengren} (Mainlander descent). List estimates are based on linear probability models; direct responses are modeled using logistic regression. Models include age, sex, education, and political identification as covariates. Error bars represent 90\% and 95\% confidence intervals. The “difference” column is \emph{List $-$ Direct} (pp) and summarizes conformity pressure. Results indicate a sizable, statistically significant gap for \textit{Benshengren} and a smaller, non-significant gap for \textit{Waishengren}, consistent with selective enforcement concentrated on the modal-origin group.}}

\end{figure}
\pagebreak

Next, in Figure 4, we addresses the same three groups in Taiwan – all respondents, Taiwanese-only identifiers, and Taiwanese-Chinese identifiers – but focuses on pride in being Chinese (\textit{zhongguo ren}). As before, the chart includes direct (self-reported) values, list-experiment estimates, and a difference score.

For all respondents, the direct measure of Chinese pride is 25 percent, whereas the list-experiment estimate is 32 percent. This yields a difference of about seven percentage points, suggesting that although Chinese identity does not dominate public discourse, some Taiwanese still acknowledge it via direct questions. The moderate gap indicates that some respondents may feel only mild social pressures discouraging expressions of Chinese identity, although the direct-list difference is statistically insignificant.

Among Taiwanese-only identifiers, both the direct (7\%) and list (16\%) measures remain low, reflecting a clear reluctance or disinterest in identifying as Chinese. The difference is not substantially large or statistically significant, reinforcing the notion that this subgroup’s minimal Chinese pride is consistent across both types of questions.
Lastly, Taiwanese-Chinese identifiers exhibit substantially higher levels of pride in being Chinese, with nearly identical direct and list-based measures (~56-57\%). Although elsewhere this group readily expresses Taiwanese identity – potentially overstating it under social expectations – their consistent and open acknowledgment of Chinese pride strongly indicates that, for this subgroup, the Chinese dimension of their identity is neither suppressed nor subject to meaningful social desirability bias. 

The absence of social desirability bias among Taiwanese-Chinese identifiers regarding pride in being Chinese highlights an important dynamic of social norms and pressures: individuals experience stronger pressures to align with dominant national identities (Taiwanese) than with identities that are less congruent with prevailing societal norms (Chinese). Consequently, pride in Chinese identity is not inflated in direct measures among Taiwanese-Chinese identifiers. Conversely, there is some indication that Taiwanese-only identifiers feel subtle pressure to suppress expressions of Chinese pride, which is what we would expect.

%%% FIGURE 4 HERE
\begin{figure}[H]
    \centering
    \includegraphics[width=1\textwidth]{Figure 4.pdf}
    \caption{Pride in Being Chinese by Identity Subgroups
    \vspace{1.5mm} \newline {\small \emph{Note}: Figure shows estimated proportions of respondents in Taiwan who express pride in being Chinese, using both direct (self-reported) and list-experiment measures. Respondents are grouped into three identity categories: all respondents, Taiwanese-only identifiers, and Taiwanese-Chinese identifiers. List estimates are based on linear probability models; direct responses are modeled using logistic regression. Estimates control for age, sex, education, and political identification. Error bars represent 90\% and 95\% confidence intervals. The “difference” column reflects the gap between list and direct responses and captures potential social desirability bias. Results suggest that Chinese identity is not suppressed among dual identifiers, who express pride across both measures, indicating that Chinese identity remains privately accepted within this subgroup despite shifting normative expectations.}}
\end{figure}

\newpage

\subsection*{Subnational Group Focus: South Korea}
Lastly, Figure 5 illustrates pride in being South Korean among native South Koreans and North Korean migrants, alongside migrants' pride in their North Korean identity. Native South Koreans express substantial pride directly (80\%) but a notably lower estimate indirectly (59\%), yielding a significant difference of 21 percentage points. This highlights strong normative expectations to express national pride publicly, resulting in considerable inflation of reported pride.

Among North Korean migrants, the direct measure of South Korean pride is extraordinarily high (96\%), even exceeding native South Koreans. However, the indirect measure dramatically drops to 28\%, producing an exceptionally large difference of nearly –70 percentage points. Conversely, migrants report very low pride in being North Korean directly (17\%) but significantly higher pride indirectly (52\%), indicating substantial stigma and social desirability bias that discourages public expressions of North Korean identity. These patterns reveal intense social pressures experienced by North Korean migrants to affirm South Korean identity publicly while suppressing their North Korean heritage.


%%% FIGURE 4 HERE
\begin{figure}[H]
    \centering
    \includegraphics[width=1\textwidth]{Figure 5.pdf}
    \caption{Pride in Being South Korean by Identity Subgroups
    \vspace{1.5mm} \newline {\small \emph{Note}: Figure shows estimated proportions of respondents in South Korea who express pride in being South Korean, using both direct (self-reported) and list-experiment measures. Respondents are grouped into three identity categories: all respondents, native-born South Koreans with weak national attachment, and North Korean-origin migrants. List estimates are based on linear probability models; direct responses are modeled using logistic regression. Estimates control for age, sex, education, and political identification. Error bars represent 90\% and 95\% confidence intervals. The “difference” column reflects the gap between list and direct responses and captures potential social desirability bias. Results show substantial pride inflation across both groups, particularly among North Korean-origin migrants, suggesting strong conformity pressures linked to dominant national identity norms in South Korea.}}
\end{figure}

\newpage

\subsection*{Results Summary}
Table 1 consolidates direct and list estimates for each subgroup; gaps are defined as \emph{List $-$ Direct} (pp), so negative values indicate inflation in direct pride reports. The table summarizes the primary national identity items only: pride in being Taiwanese (Taiwan) and pride in being South Korean (South Korea). We read the table against $H_1$–$H_4$.

Consistent with $H_1$ (within-case, subjective identity strength), weaker identifiers show larger conformity gaps than stronger identifiers in both societies. Among natives in South Korea the weak–strong contrast is pronounced ($-26.5$pp vs.\ $-11.9$pp, the latter n.s.); in Taiwan it is also clear ($-16.6$pp vs.\ $-9.0$pp, n.s.). This pattern underscores the common mechanism: weaker identifiers have more scope and incentive to overstate pride in direct answers.

For $H_2$ (cross-case magnitude along the subjective channel), gaps are uniformly larger in South Korea than in Taiwan for matched groups: All ($-20.7$pp vs.\ $-12.4$pp), Weak ($-26.5$pp vs.\ $-16.6$pp), and Strong ($-11.9$pp vs.\ $-9.0$pp; both n.s.). This supports the most similar systems design-based claim that the same mechanism operates more forcefully in South Korea’s more consolidated identity regime.

Turning to the case-specific implications, $H_3$ (Taiwan: selective, identity-contingent enforcement) receives support on both channels. \emph{Subjective channel:} Dual identifiers exhibit a large, significant gap ($-21.0$pp), while Taiwanese-only respondents do not ($-6.5$pp, n.s.), consistent with boundary positions attracting scrutiny and leaving room to credibly overstate in direct reports. \emph{Objective/ascribed channel:} At first glance, the smaller Waishengren gap ($-7.9$pp, n.s.) may seem counterintuitive if one expects majority pressure to fall on minorities. Our interpretation is that pressure is subgroup-specific: members of the modal-origin majority Benshengren face stronger expectations to perform Taiwanese pride and show a detectable gap ($-13.1$pp), whereas minority-origin respondents Waishengren face lower expectations and thus show a smaller, indistinguishable-from-zero gap. The Ben/Wai pattern is visualized in Figure~3; it sharpens, rather than contradicts, the selective-enforcement claim.

For $H_4$ (South Korea: broad, graded enforcement), the ordering is strongly graded: North Korean migrants show by far the largest gap ($-68.0$pp), followed by weak natives ($-26.5$pp), with strong natives showing no statistically significant gap ($-11.9$pp, n.s.). This aligns with the expectation that consolidated identity norms generate widespread pressure, strongest where audience scrutiny and vulnerability are highest.

Because $H_1$–$H_4$ are defined on pride in the dominant national identity, Table 1 focuses on those outcomes. Alternative pride items -- pride in being Chinese (Taiwan) and, for North Korean migrants, pride in being from North Korea -- serve to diagnose the direction of enforcement. As presented above, we find little to no suppression of Chinese pride (including among dual identifiers), reinforcing that selective enforcement operates mainly by inflating Taiwanese pride rather than suppressing Chinese pride. Among North Korean migrants in South Korea, direct reports markedly understate pride in being from North Korea relative to list estimates, mirroring the very large overstatement of South-Korean pride and underscoring broad, graded enforcement.

Overall, the evidence presented in the table fits the MSSD logic: South Korea exhibits broadly enforced, graded pressures, whereas Taiwan shows selective, boundary-sensitive pressures concentrated on dual identifiers and the modal-origin majority.

\newpage
\vspace*{\fill}
%% table below
% Requires: \usepackage{booktabs}
\begin{table}[!h]
\centering
\begin{tabular}{lrrrr}
\toprule
 & \textbf{Direct (\%)} & \textbf{List (\%)} & \textbf{Gap (pp)} & \textbf{Sig.} \\
\midrule
\multicolumn{5}{l}{\textbf{South Korea}} \\
\quad All respondents                 & 79.7 & 59.0 & $-20.7$ & **   \\
\quad Strong ID                       & 88.5 & 76.6 & $-11.9$ & n.s. \\
\quad Weak ID                         & 70.4 & 43.9 & $-26.5$ & **   \\
\quad North Korea-origin migrants              & 96.0 & 28.0 & $-68.0$ & **   \\
\addlinespace
\multicolumn{5}{l}{\textbf{Taiwan}} \\
\quad All respondents                 & 79.4 & 67.0 & $-12.4$ & **   \\
\quad Strong ID                       & 86.1 & 77.1 & $-9.0$  & n.s. \\
\quad Weak ID                         & 70.4 & 53.8 & $-16.6$ & **   \\
\quad Taiwanese-only identifiers      & 89.0 & 82.5 & $-6.5$  & n.s. \\
\quad Taiwanese--Chinese identifiers  & 70.0 & 49.0 & $-21.0$ & **   \\
\quad {Benshengren}                   & 81.3 & 68.2 & $-13.1$ & **   \\
\quad {Waishengren}                   & 67.6 & 59.7 & $-7.9$  & n.s. \\
\bottomrule
\end{tabular}
\caption{Direct vs.\ Indirect (List) Estimates of National Pride in South Korea and Taiwan
\vspace{1.5mm}\newline
{\small \emph{Note}: Table reports estimated proportions using both direct (self-reported) and list-experiment measures for the primary national identity items only (Taiwan: "proud to be Taiwanese"; South Korea: "proud to be South Korean"). Models include age, sex, education, and political identification as covariates. Significance based on 95\% confidence intervals.}}
\label{tab:gap_summary_grouped}
\end{table}
\vspace*{\fill}
\newpage

\subsection*{Additional Analysis: Issue-Linked Conformity Tests}
Here, we report the additional list of experiment items to test social norms and enforcement beyond identity performance (pride), specifically regarding support for a formal declaration of independence in Taiwan and for supporting the National Security Act (NSA) and unification in South Korea. Figures and supporting text are in Appendix E.

For the independence declaration in Taiwan (Figure E.1), we find no meaningful direct–list difference regarding support, including among Taiwanese–Chinese dual identifiers. In the same sample that shows a sizable conformity gap on Taiwanese pride, policy support for independence is not misreported. This suggests that in Taiwan’s pluralistic regime, symbolic affirmations of belonging (identity performance) carry higher reputational stakes than overt policy stances on sovereignty.

In Figure E.2, we show findings for South Korea. Here, we show that the National Security Act (NSA) has a large and statistically significant direct–list gap, indicating overstated public support consistent with conformity to security–identity norms. By contrast, views on (opposition to) unification display almost no direct–list difference, implying that unification no longer functions as an enforced identity norm. Together, the NSA/unification results map precisely onto where social monitoring is salient in South Korea’s consolidated identity regime.

These issue-linked tests corroborate the main claims presented in this paper. In Taiwan, conformity pressures focus on identity performance (pride), rather than sovereignty policy. In South Korea, pressures extend to loyalty-inflected security items (NSA) but not to legacy symbolism (unification). The additional analyses, therefore, clarify the scope conditions of enforcement: where norms are historically institutionalized, socially enforced, and symbolically charged, preference falsification is most likely to appear.

\section*{Discussion and Conclusion}
The results of the list experiments demonstrate significant variation in how social desirability pressures shape expressions of national pride among Taiwanese, native South Korean, and North Korean migrant respondents. Our analysis first established how the strength of identification with the dominant national identity influences the extent of pride inflation. In both cases, the overall findings indicate that individuals with weaker national identification are more prone to overstating their pride, but in South Korea the pressure to conform is significantly greater. In Taiwan, the general level of pride inflation is moderate. Taken together, these results provide clear support for $H_1$ (within-case identity strength), $H_2$ (cross-case magnitude), and the case-conditional expectations in $H_3$ and $H_4$. Importantly, these core tests are defined on the primary national identity outcomes: pride in being Taiwanese and pride in being South Korean.

Subsequently, we focused on specific subgroups to explore additional layers of social and political pressures faced by groups with contested or stigmatized identities. Notable differences emerge between the two cases. Those who strongly identify as Taiwanese consistently expressed pride across direct and indirect measurements, indicating genuine and socially accepted pride. However, individuals who maintain dual identities as both Taiwanese and Chinese demonstrate significant pride inflation when expressing Taiwanese identity publicly. This suggests that social expectations in Taiwan actively encourage a public alignment with Taiwanese identity, particularly among those whose attachments are less exclusive or stable. At the same time, pride in being Chinese is not suppressed among dual identifiers, reinforcing our interpretation of Taiwan as a pluralistic environment in which selective pressure operates primarily by inflating Taiwanese pride rather than by suppressing Chinese pride (cf.\ Wu \& Lin, 2024). A complementary pattern appears for paternal-origin subgroups: Benshengren (the modal-origin majority) show a detectable conformity gap, whereas Waishengren do not. Although this may seem counterintuitive if one expects pressure to fall on minorities, it is consistent with $H_3$’s selective, identity-contingent enforcement: majority-background respondents face stronger expectations to perform Taiwanese pride and thus more often overstate it in direct reports.

South Korea presents a different pattern, with considerably higher overall levels of pride inflation. Both native South Koreans and North Korean migrants show pronounced social desirability pressures to conform publicly to South Korean national identity expectations. Among native South Koreans, these pressures are especially visible among weak identifiers, whose reported pride is markedly inflated relative to their stronger counterparts, underscoring that conformity extends beyond marginalized groups. Particularly notable is the case of North Korean migrants, who exhibit extreme levels of pride inflation when expressing allegiance to South Korean identity in direct questioning, but significantly lower levels when measured indirectly through list experiments. Additionally, these migrants substantially understate their pride in being from North Korea in direct reports relative to list estimates, indicating strong stigma attached to their origin (Kim \& Oh, 2001) and mirroring the broad, graded enforcement expected under $H_4$. This further highlights how, for stigmatized minority groups, individuals self-censor their national identity to avoid social rejection (Neumann 1974; Suciu \& Culea, 2015).

These comparative findings underscore the critical role played by state-driven identity formation and state–society relations in shaping how national pride is expressed in divided societies. Despite shared structural features, South Korea’s cohesive and exclusionary identity narrative, which is deeply embedded in historical legacies of anti-communism and ethno-national unity (Lee \& Misco, 2014; Lee, 2021), fosters an environment where conformity pressures are intense, leading both marginalized groups and weak identifiers within the majority population to overstate their allegiance when asked about it directly. Taiwan’s more fluid state–society relations and evolving identity politics create a relatively more flexible environment (Tu 1996; Chu \& Lin 2001; Hur, 2022). While there is pressure to align with Taiwanese identity, especially among boundary-position (dual) identifiers, this pressure does not generally extend to outright suppression of alternative identities.

In the South Korean case, these conformity pressures are not only socially embedded but also politically institutionalized. The state itself plays an active role in maintaining national identity boundaries, particularly through legal instruments and national security discourse. As seen in the additional National Security Act (NSA) findings, national identity is not merely a matter of civic belonging but is securitized—violations or deviations are treated as potential threats to state legitimacy. This dual enforcement, by both society and state, helps explain the intensity of identity performance pressures, especially among groups already positioned at the margins of the national narrative.

More broadly, the overstatement of pride among subgroups navigating contested or marginalized identities highlights mechanisms of nation-building through social conformity in divided societies. Majority populations frequently impose pressures on minority or contested identity groups to align publicly with dominant national narratives, resulting in identity concealment and preference falsification (Kachuyevski \& Olesker, 2014; Mac Ginty, 2017). While prior work contrasts top-down enforcement in autocracies with banal nationalism in democracies (Wedeen, 1999; Dukalskis, 2020; Goode, 2021), our results clarify \emph{how} social norms endorse conformity to majority identity in democratic contexts and \emph{which} subgroups bear that pressure—boundary-position and majority-background respondents in Taiwan; weak natives and especially NK-origin migrants in South Korea.

Our study shows that social desirability biases around national identity expression are not restricted to authoritarian regimes or traditionally sensitive social categories but are also prominent in democratic contexts. In doing so, this study addresses the empirical and theoretical gap identified by Mylonas and Tudor (2021), who note that research on nationalism has disproportionately emphasized the origins and content of national identity, while paying comparatively little attention to how dominant identity narratives are maintained and enforced. By examining the conditions under which individuals falsify expressions of national pride, our findings show how identity norms are sustained through social conformity pressures, even in democratic contexts. Rather than assuming internalized consensus, we demonstrate that alignment with dominant identity narratives often reflects conformity with perceived social expectations, particularly among those with weak or stigmatized identity positions. Robustness checks for Taiwan, including secure/insecure recoding and immigrant exclusions (Appendix~E), confirm that these findings are not artifacts of subgroup coding choices.

Finally, this article underscores the importance of experimental measures to capture hidden identities and real levels of national pride. By highlighting the interaction between genuine sentiment and strategic conformity, our findings contribute to understanding how citizens in divided societies negotiate and publicly perform their national allegiances. Future work should incorporate additional divided contexts and minority populations, and further probe alternative identity items to assess whether conformity operates through \emph{inflation} of dominant-identity pride, \emph{suppression} of minority-identity pride, or both, across varying institutional settings and audience structures.


\section*{Funding}
This work was supported by the Academy of Korean Studies under Grant AKS-2023-R-018.

\section*{Ethics Statement}
This research was conducted in accordance with ethical guidelines approved by the University of Vienna Ethics Committee under decision numbers 00654 and 00997.

\nocite{*}
\section*{References}\label{sec:references}
\printbibliography[heading=none]

\end{document}
